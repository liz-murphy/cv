%%%%%%%%%%%%%%%%%%%%%%%%%%%%%%%%%%%%%%%%%%%%%%%%%%%%%%%%%%%%%%%%%%%%%%%%
%%%%%%%%%%%%%%%%%%%%%% Simple LaTeX CV Template %%%%%%%%%%%%%%%%%%%%%%%%
%%%%%%%%%%%%%%%%%%%%%%%%%%%%%%%%%%%%%%%%%%%%%%%%%%%%%%%%%%%%%%%%%%%%%%%%

%%%%%%%%%%%%%%%%%%%%%%%%%%%%%%%%%%%%%%%%%%%%%%%%%%%%%%%%%%%%%%%%%%%%%%%%
%% NOTE: If you find that it says                                     %%
%%                                                                    %%
%%                           1 of ??                                  %%
%%                                                                    %%
%% at the bottom of your first page, this means that the AUX file     %%
%% was not available when you ran LaTeX on this source. Simply RERUN  %% 
%% LaTeX to get the ``??'' replaced with the number of the last page  %% 
%% of the document. The AUX file will be generated on the first run   %%
%% of LaTeX and used on the second run to fill in all of the          %%
%% references.                                                        %%
%%%%%%%%%%%%%%%%%%%%%%%%%%%%%%%%%%%%%%%%%%%%%%%%%%%%%%%%%%%%%%%%%%%%%%%%

%%%%%%%%%%%%%%%%%%%%%%%%%%%% Document Setup %%%%%%%%%%%%%%%%%%%%%%%%%%%%

% Don't like 10pt? Try 11pt or 12pt
\documentclass[10pt]{article}

% This is a helpful package that puts math inside length specifications
\usepackage{calc}

% Layout: Puts the section titles on left side of page
\reversemarginpar

%
%         PAPER SIZE, PAGE NUMBER, AND DOCUMENT LAYOUT NOTES:
%
% The next \usepackage line changes the layout for CV style section
% headings as marginal notes. It also sets up the paper size as either
% letter or A4. By default, letter was used. If A4 paper is desired,
% comment out the letterpaper lines and uncomment the a4paper lines.
%
% As you can see, the margin widths and section title widths can be
% easily adjusted.
%
% ALSO: Notice that the includefoot option can be commented OUT in order
% to put the PAGE NUMBER *IN* the bottom margin. This will make the
% effective text area larger.
%
% IF YOU WISH TO REMOVE THE ``of LASTPAGE'' next to each page number,
% see the note about the +LP and -LP lines below. Comment out the +LP
% and uncomment the -LP.
%
% IF YOU WISH TO REMOVE PAGE NUMBERS, be sure that the includefoot line
% is uncommented and ALSO uncomment the \pagestyle{empty} a few lines
% below.
%

%% Use these lines for letter-sized paper
\usepackage[paper=letterpaper,
            %includefoot, % Uncomment to put page number above margin
            marginparwidth=1.2in,     % Length of section titles
            marginparsep=.05in,       % Space between titles and text
            margin=1in,               % 1 inch margins
            includemp]{geometry}

%% Use these lines for A4-sized paper
%\usepackage[paper=a4paper,
%            %includefoot, % Uncomment to put page number above margin
%            marginparwidth=30.5mm,    % Length of section titles
%            marginparsep=1.5mm,       % Space between titles and text
%            margin=25mm,              % 25mm margins
%            includemp]{geometry}

%% More layout: Get rid of indenting throughout entire document
\setlength{\parindent}{0in}

%% This gives us fun enumeration environments. compactitem will be nice.
\usepackage{paralist}

%% Reference the last page in the page number
%
% NOTE: comment the +LP line and uncomment the -LP line to have page
%       numbers without the ``of ##'' last page reference)
%
% NOTE: uncomment the \pagestyle{empty} line to get rid of all page
%       numbers (make sure includefoot is commented out above)
%
\usepackage{fancyhdr,lastpage}
\pagestyle{fancy}
%\pagestyle{empty}      % Uncomment this to get rid of page numbers
\fancyhf{}\renewcommand{\headrulewidth}{0pt}
\fancyfootoffset{\marginparsep+\marginparwidth}
\newlength{\footpageshift}
\setlength{\footpageshift}
          {0.5\textwidth+0.5\marginparsep+0.5\marginparwidth-2in}
\lfoot{\hspace{\footpageshift}%
       \parbox{4in}{\, \hfill %
                    \arabic{page} of \protect\pageref*{LastPage} % +LP
%                    \arabic{page}                               % -LP
                    \hfill \,}}

% Finally, give us PDF bookmarks
\usepackage{color,hyperref}
\definecolor{darkblue}{rgb}{0.0,0.0,0.3}
\hypersetup{colorlinks,breaklinks,
            linkcolor=darkblue,urlcolor=darkblue,
            anchorcolor=darkblue,citecolor=darkblue}

%%%%%%%%%%%%%%%%%%%%%%%% End Document Setup %%%%%%%%%%%%%%%%%%%%%%%%%%%%


%%%%%%%%%%%%%%%%%%%%%%%%%%% Helper Commands %%%%%%%%%%%%%%%%%%%%%%%%%%%%

% The title (name) with a horizontal rule under it
%
% Usage: \makeheading{name}
%
% Place at top of document. It should be the first thing.
\newcommand{\makeheading}[1]%
        {\hspace*{-\marginparsep minus \marginparwidth}%
         \begin{minipage}[t]{\textwidth+\marginparwidth+\marginparsep}%
                {\large \bfseries #1}\\[-0.15\baselineskip]%
                 \rule{\columnwidth}{1pt}%
         \end{minipage}}

% The section headings
%
% Usage: \section{section name}
%
% Follow this section IMMEDIATELY with the first line of the section
% text. Do not put whitespace in between. That is, do this:
%
%       \section{My Information}
%       Here is my information.
%
% and NOT this:
%
%       \section{My Information}
%
%       Here is my information.
%
% Otherwise the top of the section header will not line up with the top
% of the section. Of course, using a single comment character (%) on
% empty lines allows for the function of the first example with the
% readability of the second example.
\renewcommand{\section}[2]%
        {\pagebreak[2]\vspace{1.3\baselineskip}%
         \phantomsection\addcontentsline{toc}{section}{#1}%
         \hspace{0in}%
         \marginpar{
         \raggedright \scshape #1}#2}

% An itemize-style list with lots of space between items
\newenvironment{outerlist}[1][\enskip\textbullet]%
        {\begin{itemize}[#1]}{\end{itemize}%
         \vspace{-.6\baselineskip}}

% An environment IDENTICAL to outerlist that has better pre-list spacing
% when used as the first thing in a \section 
\newenvironment{lonelist}[1][\enskip\textbullet]%
        {\vspace{-\baselineskip}\begin{list}{#1}{%
        \setlength{\partopsep}{0pt}%
        \setlength{\topsep}{0pt}}}
        {\end{list}\vspace{-.6\baselineskip}}

% An itemize-style list with little space between items
\newenvironment{innerlist}[1][\enskip\textbullet]%
        {\begin{compactitem}[#1]}{\end{compactitem}}

% To add some paragraph space between lines.
% This also tells LaTeX to preferably break a page on one of these gaps
% if there is a needed pagebreak nearby.
\newcommand{\blankline}{\quad\pagebreak[2]}

%%%%%%%%%%%%%%%%%%%%%%%% End Helper Commands %%%%%%%%%%%%%%%%%%%%%%%%%%%

%%%%%%%%%%%%%%%%%%%%%%%%% Begin CV Document %%%%%%%%%%%%%%%%%%%%%%%%%%%%

\begin{document}
\makeheading{Liz Murphy}

\section{Contact Information}
%
% NOTE: Mind where the & separators and \\ breaks are in the following
%       table.
%
% ALSO: \rcollength is the width of the right column of the table 
%       (adjust it to your liking; default is 1.85in).
%
\newlength{\rcollength}\setlength{\rcollength}{1.85in}%
%
\begin{tabular}[t]{@{}p{\textwidth-\rcollength}p{\rcollength}}
%\href{http://www.ece.osu.edu/}%
%     {Department of Electrical and Computer Engineering} & \\
 
90 S 13th Street \#2                          & \textit{Mobile:} +1 (202) 817-5028 \\
San Jose\\
CA 95112\\
{E-mail: }\href{mailto:liz.murphy80@gmail.com} {liz.murphy80@gmail.com} \\
www: \href{http://www.lizmurphy.org/}{www.lizmurphy.org} 
\end{tabular}

%\section{Security Clearance} 
%
%Department of Defense Top Secret SCI with polygraph (expired: 2002) 

\section{Citizenship}
%
Australian (E3 Visa Status)

\section{Skills}
%
Robotics Engineer specializing in algorithms and software development for autonomous navigation for ground vehicles. C++, Python, ROS 

\section{Professional Experience}
{\textbf{Apple}}, Sunnyvale CA
\begin{outerlist}
\item[] \textit{Software Engineer, Special Projects Group} \hfill \textbf{July 2016 - Present}
\\
\end{outerlist}

\href{http://www.savioke.com}{\textbf{Savioke}}, Santa Clara CA
\begin{outerlist}
\item[] \textit{Senior Robot Engineer} \hfill \textbf{June 2014 - June 2016}
\begin{innerlist}
\item Designing, implementing and testing algorithms for localization, mapping and path planning for mobile robots in
the service industry. Our fleet of robots run 24/7 around people, they take the elevator, and have completed over 13000
deliveries to date. 
\item Personally responsible for a number of systems, including the SLAM system being used for mapping, the dock detection system, floor (elevator)
detection using barometric data, and door opening detection for delivery validation
\item Worked collaboratively on the path planning (local and global), localization and logging and introspection
systems
\item Part of a navigation software team of 3, distributed across 3 continents. 
\item Software developement in C++ and Python using the ROS framework on Linux. 
\item Experience with a range of sensors including LiDAR, 3D Depth cameras, IMU, Barometer and Wheel Odometry
\item Agile software development
\item Testing and deployment of the software to the fleet of 40+ robots.
\end{innerlist}
\end{outerlist}
\blankline

\href{http://www.gwu.edu}{\textbf{George Washington University}}, 
Washington, DC
\begin{outerlist}
\item[] \textit{Postdoctoral Scientist} \hfill \textbf{June 2013 - June 2014}
\begin{innerlist}
\item{Worked in the Robotics and Perception Group}
\item{Built a system to perform visual place recognition using a vocabulary of visual words learned on-line}
\item{Leveraged this framework to detect distinct topological places to aid long term place recognition}
\item{Integrated the system into the group's visual SLAM framework (relative SLAM)}
\end{innerlist}
\end{outerlist}
\begin{outerlist}
\item[] \textit{Adjunct Lecturer} \hfill \textbf{Spring 2014}
\begin{innerlist}
\item{CSCI4527/6547 Introduction to Computer Vision}
\end{innerlist}
\end{outerlist}
\blankline

\href{http://www.qut.edu.au}{\textbf{Queensland University of Technology}}, 
Brisbane, Australia
\begin{outerlist}
\item[] \textit{Postdoctoral Research Fellow} \hfill \textbf{January 2011 - May 2013}
\begin{innerlist}
\item{Worked in the CyberPhysical Systems Group}
\item{Under the persistent vision-only navigation project, investigated and implemented multi-hypothesis SLAM
techniques for robust navigation, made compact models of the
enviroment for long term localization and implemented a topometric approach to navigation using only stereo vision}
\item{Traversability estimation, probabilistic costmap creation and planning using the A* and D* family of heuristic
search implemented on a remote controlled car}
\item{Converted an Adept GuiaBot for use under the ROS platform, including adding additional computers and sensors to
the platform} 
\end{innerlist}
\end{outerlist}

\blankline

\href{http://www.ox.ac.uk}{\textbf{University of Oxford}}, 
Oxford, UK
%
\begin{outerlist}

\item[] \textit{Non-stipendiary Lecturer, New College}% 
        \hfill \textbf{Michaelmas 2009}
	\begin{innerlist}
	\item Undergraduate tutorial teaching for Electronics and Communications Systems.
    \end{innerlist}

\item[] \textit{Laboratory demonstrator}% 
        \hfill \textbf{January 2007 to October 2010}
\end{outerlist}

\blankline

%
\href{http://www.energy.qld.gov.au/}{\textbf{Queensland Department of Energy}}, 
Brisbane, QLD, Australia

\begin{outerlist}
\item[] \textit{Senior Performance Officer}%
        \hfill \textbf{October 2005 - August 2006}
\begin{innerlist}
\item Operational performance monitoring of energy industry participants in Queensland
\end{innerlist}
\end{outerlist}

\blankline

\href{http://www.dsto.defence.gov.au/}{\textbf{Defence Science Technology Organisation}}, 
Adelaide, Australia
\begin{outerlist}
\item[] \textit{Research Engineer}%
        \hfill \textbf{January 2003 - October 2005}
\begin{innerlist}
\item Worked on the electronic and software system designs for a new portable secure communications device used in Defence
\item Devised solutions for the repair/upgrade of legacy electronic systems for airborne mission systems
\end{innerlist}
\end{outerlist}

\section{Education}
%
\href{http://www.ox.ac.uk/}{\textbf{University of Oxford}}, 
Oxford, UK
\begin{outerlist}

\item[] DPhil, 
        \href{http://www.eng.ox.ac.uk/}
             {Engineering Science} 
        (Completed: December 2010)
        \begin{innerlist}
        \item Thesis Topic: Planning and Exploring Under Uncertainty
        \item Supervisor: 
              \href{http://www.robots.ox.ac.uk/~mobile/wikisite/pmwiki/pmwiki.php?n=People.PMN}
                   {Dr Paul Newman}
        \item Area of Study: Mobile Robotics       
        \end{innerlist}
\end{outerlist}

\blankline        

\href{http://www.eng.jcu.edu.au/}{\textbf{James Cook University}}, 
Townsville, Queensland, Australia
\begin{outerlist}

\item[] BE(Hons)-BSc, 
        \href{http://www.eng.jcu.edu.au/}
             {Computer Systems Engineering and Computer Science}, December 2002
        \begin{innerlist}
        \item Class I Honours in Engineering        
        \item{University Medallist}
        \item{Graduated with a GPA of 6.975 on a 7.0 scale}
        \end{innerlist}

\end{outerlist}

\section{Awards} 
%
\href{http://www.rhodeshouse.ox.ac.uk/}{\textbf{The Rhodes Trust}}
\begin{innerlist}
\item Rhodes Scholarship for Australia-at-Large, October 2006 - September 2009
\end{innerlist}

\blankline

\href{http://www.jcu.edu.au}{\textbf{James Cook University}}
\begin{innerlist}
\item University Medal, 2003
\end{innerlist}

\blankline

\href{http://www.deewr.gov.au/Schooling/AustralianStudentPrize/Pages/default.aspx}{Australian Government}
\begin{innerlist}
\item Australian Student Prize, 1997
\begin{innerlist}
\item awarded to the top 500 Year 12 students in Australia
\end{innerlist}
\end{innerlist}

\pagebreak

\section{Journal Publications}
Liz Murphy and Paul Newman, \textit{Risky Planning on Probabilistic Costmaps for Path Planning in Outdoor Environments}. IEEE Transactions on Robotics.  Robotics, IEEE Transactions on, Volume 29, Issue 2. 2013\\

Paul Newman, Gabe Sibley, Mike Smith, Mark Cummins, Alastair Harrison, Chris Mei, Ingmar Posner, Robbie Shade, Derik Schrter, Liz Murphy, Winston Churchill, Dave Cole and Ian Reid \textit{Navigating, Recognising and Describing Urban Spaces With Vision and Laser}. The International Journal of Robotics Research, Volume 28, Issue 11-12. 2009

\section{Conference Publications and Presentations}
Liz Murphy and Gabe Sibley, \textit{Incremental Unsupervised Topological Place Discovery}, In Proceedings of the International Conference on Robotics and Automation (ICRA), Hong Kong, 2014 (To Appear) 

\blankline

Gabe Sibley, Nima Keivan, Alonso Patron-Perez, Vincent Spinello-Mamo, Steven Lovegrove and Liz Murphy, \textit{Scalable Perception and Planning Based Control}, In Proceedings of the International Symposium of Robotic Research (ISRR), Singapore, 2013 

\blankline

Liz Murphy and Peter Corke, \textit{STALKERBOT: Learning to Navigate Dynamic Human Environments by Following People}.  In Proceedings of the Australasian Conference on Robotics and Automation, Wellington, New Zealand, 2012  

\blankline

Liz Murphy, Steven Martin and Peter Corke, \textit{Creating and Using Probabilistic Costmaps from Vehicle Experience}. In Proceedings of the IEEE/RSJ International Conference on Intelligent Robots and Systems, Vilamoura, Portugal, 2012

\blankline

Liz Murphy, Timothy Morris, Ugo Fabrizi, Michael Warren, Michael Milford, Ben Upcroft, Michael Bosse and Peter Corke, \textit{Experimental Comparison of Odometry Approaches}. In Proceedings of the International Symposium of Experimental Robotics (ISER), Quebec City, Canada, 2012

\blankline

Steven Martin, Liz Murphy and Peter Corke, \textit{Building Large Scale Traversability Maps Using Vehicle Experience}. In Proceedings of the International Symposium of Experimental Robotics (ISER), Quebec City, Canada.

\blankline

Liz Murphy, Peter Corke and Paul Newman, \textit{Choosing Landmarks for Risky Planning}. In Proceedings of the IEEE/RSJ International Conference on Intelligent Robots and Systems (IROS), San Francisco CA, 2011

\blankline

Liz Murphy and Paul Newman, \textit{Risky Planning: Path Planning over Costmaps with a Probabilistically Bounded Speed-Accuracy Tradeoff}. In Proceedings of the IEEE International Conference on Robotics and Automation (ICRA'11). Shanghai, China. 2011

\blankline

Liz Murphy and Paul Newman, \textit{Planning Most-Likely Paths from Overhead Imagery}, In Proceedings of the International Conference on Robotics and Automation (ICRA). Anchorage, AK. 2010

\blankline

Liz Murphy and Paul Newman, \textit{Using Incomplete Online Metric Maps for Topological Exploration with the Gap Navigation Tree}, In Proceedings of the International Conference of Robotics and Automation (ICRA), Pasadena, 2008

\section{Workshops}
Liz Murphy and Peter Corke. \textit{RGBD Enabled Human Centric Navigation}. In proceedings of the RGB-D: Advanced Reasoning with Depth Cameras Workshop, at the Robotics Science and Systems Conference, Sydney, July 2012

\blankline

Liz Murphy and Paul Newman, \textit{Robust Topological Exploration with the Gap Navigation Tree}, Presentation at the RSS Workshop on Topology and Minimalism in Robotics, Zurich 2008

\section{DPhil Thesis}
Liz Murphy, \textit{Planning and Exploring Under Uncertainty}, DPhil Thesis, University of Oxford, December 2010

\section{Internal Reports}
%
Liz Murphy, \textit{Autonomous Exploration in Unknown Environments}, DPhil Transfer Report, University of Oxford, Michaelmas Term 2007

\section{Professional Affiliations} 
Member of the Institution of Electrical and Electronic Engineers (IEEE)\\

%\section{Leadership}
%%
%$\bullet$ Oxford University Womens Boat Club Committee Member 2009-2010 \\
%$\bullet$ Somerville College Middle Common Room (MCR) Social Secretary 2007-08 \\
%$\bullet$ President of the South Australian Flying Disc (Ultimate Frisbee) Association (2004) \\
%$\bullet$ United Collegians Athletic Club Committee Member 2004 \\
%$\bullet$ Athletics North Queensland Athlete's Commission Chair 2002 \\
%
%\section{Sporting Achievements}
%%
%$\bullet$ Oxford Full Blue in Rowing and Athletics \\
%\hspace*{15pt} $\bullet$ Rowed for the winning Oxford University Women's Boat Club \textit{Blue Boat} against Cambridge in 2009 \& 2010\\
%\hspace*{15pt} $\bullet$ Member of the Varsity Winning Women's Blues Athletics Team 2007 \& 2008 \\
%$\bullet$ British University Sports Association (BUSA) Heptathlon Champion, 2007 \\
%$\bullet$ Placed 6th in the Open division Heptathlon at the Australian Championships, 2000, 2001, 2004 \\
%$\bullet$ Australian representative at the Oceania Athletics Championships in Townsville (1996) and Christchurch (2002)\\
%\hspace*{15pt} $\bullet$ 2002 Oceania Area Champion in the Long Jump \\
%$\bullet$ Three time (1999,2000,2002) member of the honorary Green and Gold Team named at the Australian University Championships \\
%\hspace*{15pt}$\bullet$ Australian University Heptathlon Champion (1999,2002); High Jump Champion (2000,2001) \\
%$\bullet$  Most Valuable Player in the Women's division at the 2004 Australian Ultimate Frisbee Championships in Brisbane \\

%\section{Voluntary Work}
%$\bullet$ Coach of Somerville College Womens 2nd VII Rowing Squad, 2008/09 \\
%$\bullet$  Australian Red Cross Save-A-Mate First Aid Unit, 2005 \\
%$\bullet$  Volunteer Official for Ross River Athletics Club and United Collegians Athletic Clubs (1998 - 2005) \\
%$\bullet$  Tree Planting and Rubbish removal for Conservation Volunteers Australia and other community groups (2004-2006) \\

\end{document}

%%%%%%%%%%%%%%%%%%%%%%%%%% End CV Document %%%%%%%%%%%%%%%%%%%%%%%%%%%%%
